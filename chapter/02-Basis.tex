\section{Technische Grundlagen}
Dieses Kapitel beschreibt die technischen Grundlagen und Werkzeuge, die für die Modellierung und Analyse der demografischen Entwicklung Chinas verwendet werden. Dazu gehören die Programmiersprache Julia und die verwendeten Pakete und Bibliotheken.

\subsection{Julia}
Julia ist eine Hochleistungsprogrammiersprache, die für numerische und wissenschaftliche Berechnungen entwickelt wurde. Sie zeichnet sich durch ihre Geschwindigkeit und Effizienz aus und wird häufig für die Verarbeitung großer Datenmengen verwendet. Julia ist eine Open-Source-Programmiersprache, die auf einer einfachen und intuitiven Syntax basiert. Julia wird häufig in den Bereichen Data Science, Machine Learning und wissenschaftliche Forschung eingesetzt.\\

In diesem Projekt verwenden wir Julia, um die demografische Entwicklung Chinas zu modellieren und zu analysieren.

\subsection{Pakete und Bibliotheken}
Die folgenden Julia-Pakete wurden verwendet, um die Daten abzurufen und zu verarbeiten:
\begin{itemize}
    \item \texttt{HTTP} - Wird verwendet, um HTTP-Anfragen zu senden und Antworten zu empfangen.
    \item \texttt{JSON}  - Wird verwendet, um JSON-Daten zu verarbeiten.
    \item \texttt{Plots} - Wird verwendet, um Diagramme und Grafiken zu erstellen.
\end{itemize}


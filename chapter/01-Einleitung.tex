
\section{Einleitung}
China ist das bevölkerungsreichste Land der Welt und spielt eine zentrale Rolle in der globalen Demografie. In den letzten Jahrzehnten hat das Land jedoch eine signifikante demografische Verschiebung erlebt, die durch eine alternde Bevölkerung und eine niedrige Geburtenrate gekennzeichnet ist. Die ein Kind Politik, die von 1979 bis 2015 in China in Kraft war, hat zu einem Ungleichgewicht zwischen den Geschlechtern und einer alternden Bevölkerung geführt. Lebenshaltungskosten und wirtschaftliche Unsicherheit haben auch dazu beigetragen, dass viele Paare sich gegen Kinder entscheiden. Diese demografischen Herausforderungen könnten langfristige Auswirkungen auf die wirtschaftliche und soziale Entwicklung Chinas haben.\\  

Unter berücksichtigung dieser Aspekte ist es von entscheidender Bedeutung, die demografische Entwicklung Chinas zu analysieren und Prognosen für die Zukunft zu erstellen.\\

Dieses Projekt zielt darauf ab, die demografische Entwicklung Chinas bis zum Jahr 2100 zu analysieren und zu modellieren. Dazu verwenden wir die Programmiersprache Julia, die sich durch ihre Leistungsfähigkeit und Effizienz bei der Verarbeitung großer Datenmengen auszeichnet. 
\\


\subsection{Problemstellung}
China steht vor einer demografischen Herausforderung, die tiefgreifende Auswirkungen auf die wirtschaftliche und soziale Struktur des Landes hat. Die Hauptprobleme sind:


\begin{enumerate}
\item Sinkende Geburtenrate: Die Geburtenrate in China ist seit Jahrzehnten rückläufig und liegt mittlerweile weit unter dem für den Bevölkerungsersatz notwendigen Niveau.
\item Alternde Bevölkerung: Der Anteil älterer Menschen in der chinesischen Gesellschaft nimmt stetig zu, was zu einer erhöhten Belastung des Rentensystems und des Gesundheitswesens führt.
%\item Niedrige Migrationsrate: Die Netto-Migration kann den Bevölkerungsrückgang nicht ausgleichen, da China historisch gesehen wenig Zuwanderung verzeichnet.
%\item Ungleichmäßige regionale Verteilung: Die Bevölkerungsdichte in China variiert stark zwischen den städtischen und ländlichen Gebieten, was zu sozialen und wirtschaftlichen Disparitäten führt.
\item Ein-Kind-Politik: Obwohl die Ein-Kind-Politik in China offiziell abgeschafft wurde, hat sie langfristige Auswirkungen auf die Bevölkerungsstruktur des Landes.
\item Zu hohe Lebenshaltungskosten: Das verfügbare Einkommen und die Lebenshaltungskosten haben einen direkten Einfluss auf die Familienplanung und die Geburtenrate in China.
\end{enumerate}
Diese Probleme führen zu einem erwarteten Bevölkerungsrückgang, der die wirtschaftliche Stabilität und das soziale Wohlergehen Chinas bedrohen könnte.

\subsection{Methodik und Strategie}

Die für diese Hausarbeit verwendeten Daten stammen aus der WorldPop Datenbank, welches eine umfassende Sammlung von Bevölkerungsdaten aus verschiedenen Quellen enthält. Diese Datenbank bietet eine API, die es ermöglicht, Bevölkerungsdaten für verschiedene Regionen und Zeiträume abzurufen.\\


Anhand dieser Datenquellen können wir die demografische Entwicklung Chinas in den letzten Jahrzehnten analysieren und Prognosen für die Zukunft erstellen.  Diese Analyse wird folgende Punkte umfassen:
\begin{itemize}
\item Das Aufstellen der Geburten- und Sterberaten der letzten 60 Jahre
\item Die Berechnung der Wachstumsrate der Bevölkerung
\item Prognosen für die Bevölkerungsentwicklung bis 2100
\end{itemize}


Julia eignet sich in unserem Fall besonders gut, zur Bearbeitung und Analyse großer Datenmengen. Die Programmiersprache bietet eine einfache Syntax und eine umfangreiche Sammlung von Bibliotheken und Paketen, die für die Datenverarbeitung und Modellierung benötigt werden.\\
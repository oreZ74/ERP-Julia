
\section{Einleitung}
China ist das bevölkerungsreichste Land der Welt und spielt eine zentrale Rolle in der globalen Demografie. In den letzten Jahrzehnten hat das Land jedoch eine signifikante demografische Verschiebung erlebt, die durch eine alternde Bevölkerung und eine niedrige Geburtenrate gekennzeichnet ist. Dieses Projekt zielt darauf ab, die demografische Entwicklung Chinas bis zum Jahr 2100 zu modellieren und zu analysieren. Dazu verwenden wir die Programmiersprache Julia, die sich durch ihre Leistungsfähigkeit und Effizienz bei der Verarbeitung großer Datenmengen auszeichnet.
\\

\section{Problemstellung}

China steht vor einer demografischen Herausforderung, die tiefgreifende Auswirkungen auf die wirtschaftliche und soziale Struktur des Landes hat. Die Hauptprobleme sind:


\begin{enumerate}
\item Sinkende Geburtenrate: Die Geburtenrate in China ist seit Jahrzehnten rückläufig und liegt mittlerweile weit unter dem für den Bevölkerungsersatz notwendigen Niveau.
\item Alternde Bevölkerung: Der Anteil älterer Menschen in der chinesischen Gesellschaft nimmt stetig zu, was zu einer erhöhten Belastung des Rentensystems und des Gesundheitswesens führt.
%\item Niedrige Migrationsrate: Die Netto-Migration kann den Bevölkerungsrückgang nicht ausgleichen, da China historisch gesehen wenig Zuwanderung verzeichnet.
%\item Ungleichmäßige regionale Verteilung: Die Bevölkerungsdichte in China variiert stark zwischen den städtischen und ländlichen Gebieten, was zu sozialen und wirtschaftlichen Disparitäten führt.
\item Ein-Kind-Politik: Obwohl die Ein-Kind-Politik in China offiziell abgeschafft wurde, hat sie langfristige Auswirkungen auf die Bevölkerungsstruktur des Landes.
\item Zu hohe Lebenshaltungskosten: Das verfügbare Einkommen und die Lebenshaltungskosten haben einen direkten Einfluss auf die Familienplanung und die Geburtenrate in China.
\end{enumerate}
Diese Probleme führen zu einem erwarteten Bevölkerungsrückgang, der die wirtschaftliche Stabilität und das soziale Wohlergehen Chinas bedrohen könnte.

\section{Methodik und Strategie}

Die für diese Hausarbeit verwendeten Daten stammen aus verschiedenen Quellen, darunter:
\begin{itemize}
\item Bevölkerungsstatistiken der Vereinten Nationen (UN)
\item National Bureau of Statistics of China (NBS)
\item WorldPop Datenbank
\end{itemize}

Anhand dieser Datenquellen können wir die demografische Entwicklung Chinas in den letzten Jahrzehnten analysieren und Prognosen für die Zukunft erstellen.  Diese Analyse kann folgende Punkte umfassen:
\begin{itemize}
\item Vorhersage des Bevölkerungsrückgangs
\item Veränderung der Altersstruktur
\item Auswirkungen auf die wirtschaftliche und soziale Struktur
\end{itemize}
Die Verwendung von Julia zur Modellierung und Analyse der demografischen Entwicklung Chinas bis 2100 bietet eine leistungsstarke und effiziente Methode, um komplexe Daten zu verarbeiten und aussagekräftige Ergebnisse zu erzielen.